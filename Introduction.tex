\section{Introduction}
It is well established that Landsat and other remotely sensed imagery can be used to study landuse change (e.g. \cite{Seto et al., 2002-}; Gaughan et al., 2009), agriculture health (e.g. Anderson et al., 2012), the impacts of climate change (e.g. Roy et al., 2014), and many other physical characteristics of the earth's surface.  
Less emphasis has been placed on using older Landsat data to apply remote sensing analyses to historical and political questions. Much as current satellite information can be used to track destruction in Syria and genocide in Myanmar (e.g.Jaafar and Woertz 2016, Hassan et al. 2018, Witmer 2015) remote sensing techniques can be applied to older earth observation data to enhance our understanding of political situations of the past.  This paper will illustrate the benefits and drawbacks of using older, relatively low-resolution satellite imagery from Landsats 1-3 to fill in gaps in the historical record.
The information extracted from these images will enhance our understanding of what was (literally) happening “on the ground” during times when very little information was available to those outside of Cambodia’s borders.


Main text paragraph. Citing a journal paper \cite{ref-journal}. And now citing a book reference \cite{ref-book}.


Main text paragraph.
